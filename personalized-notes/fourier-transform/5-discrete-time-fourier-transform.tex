\section{Discrete Time Fourier Transform}
There are times when we have finite amount of points on a function, 
and we wanted to curve fit the data points with cosine and sine functions.
This is the problem that the Discrete Time Fourier Transform (DTFT) and the Discrete Fourier Transform (DFT) solved. 
The difference between them is that DTFT outputs a continuous function of frequency while DFT outputs a discrete one. 

Due to that time is discrete in DTFT, we have to replace the integral with summation: 
\begin{equation}
    F(\omega) = \sum_{t=-\infty}^{+\infty} f[t] e^{-\mathrm{i}\omega t} 
    \label{equ:DTFT}
\end{equation}

From \eqref{equ:DTFT}, we can tell that $F(\omega)$ is a function of sum of cosine and sine functions. 
Which means that $F(\omega)$ is a periodic function. 
Compare the fourier series with the DTFT, it again reveals the symmetrical property of the Fourier transform. 

On the other hand, frequency is continuous, which we could use the same equation for 
the inverse transformation as the Fourier transform derived for the continuous Fourier transform, 
only with a limited amount of changes. 
Time is discrete, which means that we have to limit/round the input of the function $f(t)$ to integers. 
To do so, simply replace $f(t)$ with $f[t]$ would be fine. 
On top of that, due to that $F(\omega)$ is a periodic function, 
setting any interval of $2\pi$ as the limits of the integral does exactly what we wanted it to do 
(it diverges if we take the integral between the interval $(-\infty,\infty)$). 
We can get the equation for 
the inverse discrete time Fourier transform (IDTFT):
\begin{equation}
    f[t] = \frac{1}{2\pi} \int_{2\pi} F(\omega) e^{\mathrm{i}\omega t} \mathrm{d}\omega
    \label{equ:IDTFT}
\end{equation}
