\section{Laplace Transform}
The Fourier transform only works with functions that do not diverge to infinity as the inputs approach infinity. 
Which means that it does not work with functions such as exponential or polynomials. The Laplace Transform is a 
generalized transformation of the Fourier transform. It can transform functions that increase as rapidly as 
exponential functions, and functions that grow slower. It expanded the set of functions that the transformation 
can be applied. However, it does not included functions that increase faster than exponential functions, such as the 
Gamma function, or double exponential functions ($f(x)=a^{(b^x)}$), etc... Another limitation to the use of the 
Laplace transform is that the function is defined for only positive values of input ($t\geq 0$). 

\indent Let's say the function, $f(t)$, that we wanted to transform is bounded by $e^{bt}$, 
where $a$ is a positive constant. We can multiply this function by a term 
which decays faster than $e^{bt}$. That term can be $e^{-at}$, where $b\geq a$. 
$$ g(t) = f(t)e^{-at} $$
We now have a function that converges as time approaches infinity.
If time is negative, it does not converges, but we do not need to consider negative 
values of time in most cases. We can apply Fourier transform to this new function. 
Let $s=a+\mathrm{i}\omega$: 

$$\begin{aligned}
    G(s) &= \int_{0}^{\infty} 
                    f(t)e^{-at}e^{-\mathrm{i}\omega t} \mathrm{d}t  \\
    &= \int_{0}^{\infty} f(t)e^{-st} \mathrm{d}t
\end{aligned}$$

Even though time has to be greater than 0, the frequency can still be negative. By substituting $s=a+\omega\mathrm{i}$, 
the inverse transform is:
$$\begin{aligned}
    g(t) &= \frac{1}{2\pi} \int_{-\infty}^{\infty} 
        G(s)e^{\mathrm{i}\omega t} \mathrm{d}\omega  \\
    f(t) &= \frac{1}{2\pi} \int_{-\infty}^{\infty} 
            G(s)e^{t(a+\mathrm{i}\omega)} \mathrm{d}\omega  \\
    f(t) &= \frac{1}{2\pi\mathrm{i}} \int_{a-\infty}^{a+\infty} G(s)e^{st} \mathrm{d}s
\end{aligned}$$

Which can be written as:
\begin{equation}\begin{cases}
    \mathcal{L}\{f(t)\} = F(s) = \displaystyle\int_{0}^{\infty} f(t)e^{-st} \mathrm{d}t     \\
    \mathcal{L}^{-1}\{F(s)\} = f(t) = \frac{1}{2\pi\mathrm{i}} \displaystyle\int_{a-\infty}^{a+\infty} F(s)e^{st} \mathrm{d}s
\end{cases}
\label{equ:laplace_transfrom}
\end{equation}
