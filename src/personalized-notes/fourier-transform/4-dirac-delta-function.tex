\section{Dirac Delta Function}
Since the function constructed only with the analytical discrete data only has definitions on some points, 
we can use the Dirac Delta Function:
\begin{align*}
\delta(t) = \begin{cases}
    +\infty     &\text{if $t = 0$,}    \\
    0           &\text{if $t \neq 0$.}
\end{cases}
\end{align*}
In addition, 

$$\int_{-\infty}^{+\infty} \delta(t) \mathrm{d}t = 1$$

Sampling of a signal occurs every time interval, while at times between two sampling occurs, 
the sampling function of continuous time that we get is 0. 
This means that the constructed sampling function has a very similar property with impulse response functions.
If we have to sample the continuous time function, $f(t)$, 
we can get a sampling function of continuous time using the Dirac Delta Function (impulse response function), 
$x[n]$ where $n$ is the time of the sampling occurs:
$$ x[n]= f(t)\delta(t-n) $$

Take samples repeatedly with a sampling time gap of $T_s$, we would get: 
$$ x_p(t)= \sum_{n=-\infty}^{+\infty} f(t)\delta(t-nT_s) $$

For the function in frequency form, we have to derive a impulse frequency function. 
We cannot simply use $F(\omega)\delta(\omega - \theta)$, because the frequency is symmetrical. 
Consider the function $$ p(t) = \cos(\omega_0t) $$ Then after Fourier transform 
$$ \mathcal{F}\{p(t)\} = P(\omega) = \pi\delta(\omega-\omega_0) + \pi\delta(\omega+\omega_0)$$

We now have the impulse frequency function, $P(t)$
$$ X_p(\omega) = \frac{1}{2\pi} \int_{-\infty}^{+\infty} F(\omega)P(\omega-\theta) \mathrm{d}\theta $$

For any functions with discrete frequencies, they must be periodic. 
Because the frequency approaches zero for non-periodic functions, 
and we could not approach zero with discrete numbers.



