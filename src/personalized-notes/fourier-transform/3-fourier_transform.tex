\section{Fourier Transform (Fourier Integral)}
For continuous periodic functions, we had showed the transform by using the Fourier series, 
we are able to describe the continuous periodic function using amplitude and phase shift instead of time. 
Not only that, the process of conversion was also showed. 

Nevertheless, the Fourier series can only deal with periodic functions and 
it cannot pick out amplitude and phase shift information of continuous frequencies. 
This is the problem that Fourier transform solves.
The Fourier integral solves both problems by setting the cycle to $-\infty$ to $\infty$, 
and expand the definition of $n$ to the real number set. 
The entire function becomes one cycle, resulting in the base frequency, 
$\omega_0$ to approach $0$. Due to the fact that $\frac{1}{T}=\frac{\omega_0}{2\pi}$, 
and $\omega_0$ equals to the difference between the two nearby frequency($n\omega_0$ and $(n+1)\omega_0$) 
in the Fourier series. 

The equation for the Fourier series
$$
    f(t) 
    = \sum_{n=-\infty}^{\infty} \left[ e^{\mathrm{i}n\omega_0t} \cdot
    \frac{\omega_0}{2\pi}\int^{T} f(t)e^{-\mathrm{i}n\omega_0t} \mathrm{d}t\right]
$$
becomes:
$$\begin{aligned}
    f(t) 
    &= \lim_{\omega_0 \to 0} \sum_{n=-\infty}^{\infty} \left[ e^{\mathrm{i}n\omega_0 t} \cdot
    \frac{\omega_0}{2\pi}\int_{-\infty}^{\infty} f(t) e^{-\mathrm{i}n\omega_0 t} \mathrm{d}t \right]   \\
    &= \int_{-\infty}^{\infty}  \left[ e^{\mathrm{i}\omega t} \cdot
    \frac{\mathrm{d}\omega}{2\pi}\int_{-\infty}^{\infty} f(t) e^{-\mathrm{i}\omega t} \mathrm{d}t \right] 
\end{aligned}$$

Re-write the above equation properly:

\begin{equation}
    f(t) 
    = \frac{1}{2\pi} \int_{-\infty}^{\infty} \left[ 
    \int_{-\infty}^{\infty} f(t) e^{-\mathrm{i}\omega t} \mathrm{d}t \right] e^{\mathrm{i}\omega t} \mathrm{d}\omega
    \label{equ:fully_expanded_fourier_integral}
\end{equation}

\indent Since the frequency became continuous, so the coefficients (what used to be $F_n$) 
became a function of frequency, or:

\begin{equation}
    F(\omega) = \int_{-\infty}^{\infty} f(t) e^{-\mathrm{i}\omega t} \mathrm{d}t
    \label{equ:forward_fourier_integral}
\end{equation}

\indent We get the equation for the inverse Fourier transform by subbing 
{\eqref{equ:forward_fourier_integral}} into {\eqref{equ:fully_expanded_fourier_integral}}:

\begin{equation}
    f(t) = \frac{1}{2\pi} \int_{-\infty}^{\infty} F(\omega) e^{\mathrm{i}\omega t} \mathrm{d}\omega
    \label{equ:inverse_fourier_integral}
\end{equation}

\indent The above equations transform between function of time and function of frequency. 
Equation {\eqref{equ:forward_fourier_integral}} is called the Forward Fourier Transform. 
It transforms the function about time into the function about frequency. On the other hand, 
equation {\eqref{equ:inverse_fourier_integral}} does the opposite. It is called the Reverse Fourier Transform, 
and it transforms the function about frequency into the function about time. 

\indent Unlike the Fourier Seires, the coefficient $\frac{1}{2\pi}$ which corresponds 
to $\frac{1}{T}$ in the Fourier Series had become a part of the {\eqref{equ:inverse_fourier_integral}} instead of 
{\eqref{equ:forward_fourier_integral}}. 
This difference does not really matter other than scaling the Fourier coefficients by a constant, 
since we don't really care about the unit of the Fourier coefficients. 
But it is still important to specify the definition of the equations to avoid any mistakes. 
We would stick with the ones that are more commonly used.

\indent The process of transformation can be written down as 
$f(t) \stackrel{\mathcal{F}}{\longleftrightarrow} F(\omega)$.

\indent The Continuous Time Fourier Transform allows the transformation of continuous non-periodic functions to be done. 
With the help of the Fourier Series, picking out frequency information of any continuous functions becomes possible.

